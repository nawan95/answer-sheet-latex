\documentclass[answer, 12pt]{exam}

% Mengubah Solution menjadi Jawaban
\renewcommand{\solutiontitle}{\noindent\textbf{Jawaban:}\par\noindent}


%%%%%%%%%%%%%%%%%%%%%%%%%%%%%%%%%%%%%%%%%%%%%%%%%%%%%%%%%%%%%%%%%%%%%%%%%%%%%%%%%%%%%%%%%%%%%%%%%%%%%
%%%%%%%%%% PAKET DI BAWAH INI KECUALI LIPSUM DIBUTUHKAN, PASTIKAN PAKET BERIKUT TERINSTALL %%%%%%%%%%
%%%%%%%%%%%%%%%%%%%%%%%%%%%%%%%%%%%%%%%%%%%%%%%%%%%%%%%%%%%%%%%%%%%%%%%%%%%%%%%%%%%%%%%%%%%%%%%%%%%%%
\usepackage{apacite} % mengatur gaya kutipan, dalam hal ini menggunakan gaya APA
\usepackage{lipsum} % opsional, untuk membuat teks lorem ipsum, gunakan perintah \lipsum untuk membuat 
%teks lorem ipsum
\usepackage[indonesian]{babel}
% \usepackage{newtxtext}
%%%%%%%%%%%%%%%%%%%%%%%%%%%%%%%%%%%%%%%%%%%%%%%%%%%%%%%%%%%%%%%%%%%%%%%%%%%%%%%%%%%%%%%%%%%%%%%%%%%%%
%%%%%%%%%% PAKET DI ATAS KECUALI LIPSUM DIBUTUHKAN, PASTIKAN PAKET DI ATAS TERINSTALL %%%%%%%%%%%%%%%
%%%%%%%%%%%%%%%%%%%%%%%%%%%%%%%%%%%%%%%%%%%%%%%%%%%%%%%%%%%%%%%%%%%%%%%%%%%%%%%%%%%%%%%%%%%%%%%%%%%%%


%%%%%%%%%%%%%%% MEMBUAT HEADER %%%%%%%%%%%%%%%%%%
\header{\bfseries Filsafat Ilmu\\Amika Wardana S.Sos., M.A. Ph.D}{}{\bfseries Ujian Akhir Semester\\ 
\today} % membuat header
\headrule % membuat garis di header
%%%%%%%%%%%%%%%%%%%%%%%%%%%%%%%%%%%%%%%%%%%%%%%%%

%%%%%%%%%%%%%%% MEMBUAT FOOTER %%%%%%%%%%%%%%%%%%
\footer{\bfseries Nawan Tabah Pangestu}{\bfseries Halaman \thepage\ dari \numpages}{\bfseries 
Pendidikan Sosiologi (B)} % membuat footer
\footrule % membuat garis di footer
%%%%%%%%%%%%%%%%%%%%%%%%%%%%%%%%%%%%%%%%%%%%%%%%%


%%%%%%%%%%%%%%%%%% SILAKAN MASUKKAN IDENTITAS SEPERTI NAMA, NIM, DAN PROGRAM STUDI %%%%%%%%%%%%%%%%%%
\newcommand{\nama}{Nawan Tabah Pangestu}       %%%%%%%%%%%  MASUKKAN NAMA
\newcommand{\NIM}{21413244035}                 %%%%%%%%%%%  MASUKKAN NOMOR INDUK MAHASISWA
\newcommand{\prodi}{Pendidikan Sosiologi / B}  %%%%%%%%%%%  MASUKKAN NAMA PRODI DAN KELAS
%%%%%%%%%%%%%%%%%%%%%%%%%%%%%%%%%%%%%%%%%%%%%%%%%%%%%%%%%%%%%%%%%%%%%%%%%%%%%%%%%%%%%%%%%%%%%%%%%%%%%


%%%%%%%%%%%%%%%%%%%%%%%%%%%%%%%%%%%%%%%%%%%%%%%%%%%%%%%%%%%%%%%%%%%%%%%%%%%%%%%%%%%%%%%%%%%%%%%%%%%%%
%%%%%%%%%%%%%%%%%%%%%%%%%%%%%%%%%%%%%%%%%%%%% BAGIAN ISI %%%%%%%%%%%%%%%%%%%%%%%%%%%%%%%%%%%%%%%%%%%%
%%%%%%%%%%%%%%%%%%%%%%%%%% BAGIAN INI BERISI INPUT PERTANYAAN BESERTA JAWABAN %%%%%%%%%%%%%%%%%%%%%%%
%%%%%%%%%%%%%%%%%%%%%%%%%%%% FILE SOAL DAN JAWABAN TERDAPAT DI FOLDER "SOAL" %%%%%%%%%%%%%%%%%%%%%%%%
%%%%%%%%%%%%%%%%%%%%%%%%%%%%%%%%%%%%%%%%%%%%%%%%%%%%%%%%%%%%%%%%%%%%%%%%%%%%%%%%%%%%%%%%%%%%%%%%%%%%%

\begin{document}
		\begin{center}
	\huge{Ujian Akhir Semester} \\
	\vspace{2mm}
	\Large{Filsafat Ilmu}
\end{center}

\vspace{5mm}
\hrule
\vspace{1mm}
\hrule

\vspace{3mm}
\begin{tabular}{ll}
	Nama & {\nama} \\
	Nomor Induk Mahasiswa & {\NIM} \\
	Prodi / Kelas & {\prodi} \\
\end{tabular}
\vspace{3mm}

\hrule
\vspace{1mm}
\hrule
\vspace{10mm}
	
	\begin{questions}
		\question
\lipsum[1]
\begin{solution}
	Aliran postivisme dan konstruktivisme adalah dua aliran dalam filsafat ilmu sosial yang sangat berbeda. Dalam hal pandangan tentang realitas sosial, positivisme memandang bahwa realitas sosial berjalan sesuai dengan hukum yang berlaku dan hanya melihat fakta-fakta empiris semata \cite{armstrong_positivism_2013}.
	Dengan kata lain, positivisme melihat realitas sosial tanpa melihat konteks sosial, motivasi, dan nilai yang ada di masyarakat yang mendasarinya.
	Dalam kasus misalnya fenomena bunuh diri, positivisme tidak melihat apa motivasi seseorang bunuh diri, tapi melihat penyebab bunuh diri dengan mencari fakta-fakta empiris kemudian disusun menjadi rantai logika hingga membentuk kausalitas; mayoritas kasus bunuh diri terjadi pada latar belakang sosial-ekonomi bawah, tidak memiliki pendidikan tinggi, pekerja serabutan, dan memiliki hutang.
	Postivisme kemudian mengambil kesimpulan bahwa bunuh diri disebabkan karena kondisi ekonomi dan lilitan hutang.
	Pendekatan positivisme berhenti di sini, ia tidak melihat konteks atau motivasi lain.
	Kenapa mereka miskin? Kenapa mereka terlilit hutang? Apa motivasi individu untuk mengakhiri hidup?
	Menurut aliran postivisme, hal itu tidak penting karena postivisme menolak spekulasi.
	Di sini dapat dilihat logika kausalitas dalam positivisme bersifat linear.
	Dengan kata lain, postivisme mengabaikan fakta bahwa realitas sosial sangat kompleks dan tidak bisa dipahami dengan logika kausalitas yang linier.
	Jika aliran positivisme melihat realitas sosial berjalan sesuai dengan hukum yang ada dan memiliki sebab-akibat yang pasti, konstruktivisme melihat bahwa realitas sosial bersifat subjektif, dan karenanya tidak ada fakta empiris dalam realitas sosial.
	
	Dalam hal fokus kajian, positivisme memiliki fokus kajian pada hasil-hasil observasi yang diinterpretasi dengan logika dan data-data empiris atau matematis dan menolak spekulasi-spekulasi teori dan hal-hal yang bersifat metafisik.
	Hal ini karena positivisme memandang bahwa informasi atau pengetahuan didapat melalui pengalaman sensori seperti observasi yang kemudian diinterpretasikan menggunakan akal dan logika \cite{macionis_sociology_2011}.
	Sedangkan konstruktivisme memiliki fokus kajian ada pada interaksi manusia dengan manusia yang lain dan dengan lingkungan.
	Hal ini karena konstruktivisme melihat pengetahuan merupakan hasil dari konstruksi oleh manusia dengan cara memberikan arti terhadap fakta atau data yang ada \cite{hinchey_chapter_2010}.
	Quantitative studies of social systems, underpinned by a positivist paradigm
	Seperti yang telah disinggung sebelumnya bahwa aliran positivisme dan konstruktivisme sangatlah berbeda.
	Pertanyaan yang muncul kemudian adalah apakah implikasi perbedaan konstruktivisme dan positivisme dalam penelitian ilmiah?
	Implikasi yang paling menonjol dari kedua aliran tersebut dalam penelitian ilmiah adalah metode penelitian yang digunakan.
	Metode penelitian kuantitatif didukung oleh aliran positivisme \cite{armstrong_positivism_2013}, sehingga penelitian ilmiah yang menggunakan paradigma positivisme cenderung menggunakan metode penelitian kuantitatif dengan analisis data alih-alih menggunakan metode kualitatif.
	Hal ini tidak mengherankan mengingat metode penelitian kuantitatif merupakan metode penelitian yang didasarkan pada pengumpulan dan analisis data-data yang bersifat empiris sesuai dengan aliran positivisme.
	Sedangkan penelitian ilmiah yang menggunakan pendekatan konstruktivisme cenderung menggunakan metode penelitian kualitatif \cite{mertens_research_2015}.
\end{solution}
		
			\question
\lipsum[2]
\begin{solution}
	Dalam penelitian ilmu sosial terdapat dua metode atau pendekatan yaitu penelitian kuantitatif dan kualitatif.
	Dua metode dalam penelitian ilmu sosial ini memiliki landasan filosofis, tujuan, dan prinsip-prinsip yang berbeda satu sama lain.
	
	Metode penelitian kuantitatif sendiri dipercaya merupakan pengaruh dari aliran filsafat positivisme \cite{ginting_filsafat_2008,mertens_research_2015,kothari_research_2013}.
	Filsafat positivisme sendiri didasarkan pada rasionalis dan empirisisme yang berdasarkan pada pemikiran Aristoteles, Auguste Comte, dan Immanuel Kant \cite{mertens_research_2015}.
	Metode kuantitatif menghendaki obyek yang teramati dan terukur untuk mencari hubungan atau pengaruh satu atau beberapa variabel terhadap variabel yang lain dengan menggunakan data-data yang bersifat empiris \cite{ginting_filsafat_2008}.
	Tujuan dalam penelitian kuantitatif yaitu `menjelaskan' \cite{ginting_filsafat_2008}.
	`Menjelaskan' yang di maksud adalah penelitian kuantitatif bertujuan untuk menjelaskan sebab-akibat satu atau lebih variabel dengan variabel yang lain berdasarkan fakta-fakta empiris. Sedangkan penelitian kualitatif memiliki tujuan `memahami'. Dengan kata lain, penelitian kualitatif bertujuan untuk memahami realitas yang ada dengan mengembangkan pengertian berdasarkan hasil kesimpulan yang diambil secara induktif.
	
	Metode penelitian kuantitatif memiliki prinsip-prinsip yang didasarkan pada aliran positivisme yaitu \cite{crossan_research_2003}:
	\begin{description}
		\item[Methodological] Semua penelitian harus menggunakan cara-cara yang terstruktur.
		\item[Value-freedom] Pengambilan keputusan dalam penelitian harus didasarkan pada kriteria objektif alih-alih kepercayaan dan kepentingan tertentu.
		\item[Causality] Tujuan penelitian haruslah untuk menemukan atau mengidentifikasi dan menjelaskan hubungan sebab-akibat dan hukum fundamental yang menjelaskan perilaku manusia.
		\item[Operationalisation] Konsep harus dibuat sedemikian rupa sehingga fakta dapat dianalisis dengan metode kuantitatif.
		\item[Independence] Peran peneliti independen dengan subjek yang diteliti.
		\item[Reductionism] Sebuah masalah akan lebih mudah dipahami jika masalah tersebut disederhanakan.
	\end{description}
	
	Penelitian kualitatif sendiri memiliki landasan filosofis pada konstruktivisme \cite{ginting_filsafat_2008,mertens_research_2015}.
	Penelitian kualitatif lahir dan berkembang dari tradisi ilmu-ilmu sosial di Jerman yang sarat diwarnai pemikiran filsafat ala platonik sebagaimana tercermin dalam pemikiran Kant maupun Hegel.
	Misalnya dalam pernyataan Hegel yang terkenal yakni `segala sesuatu yang riil adalah rasional dan segala sesuatu yang rasional adalah riil'. Paradigma penelitian kualitatif menuntut agar obyek yang diteliti untuk tidak dilepaskan dari konteksnya.
	Dengan kata lain metode penelitian kuantitatif bertolak belakang dengan metode penelitian kuantitatif dalam arti penelitian ini mengasumsikan bahwa realitas empiris terjadi bukan tanpa konteks sosio-kultural dan saling berkaitan satu sama lain.
	
	Sedangkan penelitian kualitatif memiliki prinsip-prinsip diantaranya.
	\begin{description}
		\item[Realitas Majemuk] Pendekatan kualitatif tidak hanya melihat realitas sebagai satu melainkan sebagai hal yang majemuk dan merupakan hasil dari konstruksi dalam pengertian holistik.
		\item[Interaktif] Berbeda dengan penelitian kuantitatif yang memandang bahwa peneliti dan subyek yang diteliti bersifat independen, dalam penelitian kualitatif melihat peneliti dan subyek penelitian sebagai bagian yang tidak terpisahkan.
		\item[Idiographic Statements] Penelitian kualitatif tidak lepas dari konteks waktu.
		\item[Tidak Bebas Nilai] Tidak seperti penelitian kuantitatif yang bebas nilai, penelitian kualitatif sebaliknya melihat sesuatu tidak bebas oleh nilai.
	\end{description}
	
	Salah satu contoh relevan penelitian kualitatif adalah dalam penelitian yang meneliti tentang transgender misalnya, menggunakan metode kualitatif membuat penelitian tersebut dilakukan dengan instrumen wawancara di mana peneliti terlibat dan berinteraksi langsung dengan subyek penelitian.
	
\end{solution}

		
			\question
Penelitian dalam rangka pengembangan ilmu pengetahuan ilmiah khususnya dalam Ilmu-Ilmu Sosial selalu membawa dampak negatif (baik diperkirakan dan diantisipasi maupun tidak diinginkan) terhadap pihak-pihak yang terlibat secara langsung maupun tidak langsung. 
Berikan contoh ilustrasi dampak negatif dari sebuah penelitian sosial; disertai dengan upaya untuk meminimalisasi/mengantisipasinya.
\begin{solution}
	Salah satu ilustrasi dampak negatif dari sebuah penelitian sosial terhadap pihak-pihak yang terlibat dapat dilihat dari \textit{The Milgram experiment(s)}, sebuah penelitian psikologis sosial yang dilakukan oleh Stanley Milgram dari Universitas Yale pada 1961.
	Penelitian ini bertujuan untuk mencari tahu sampai sejauh mana orang-orang akan mematuhi figur yang memiliki otoritas ketika disuruh untuk melakukan hal yang berlawanan dengan hati nurani dan berbahaya \cite{milgram_behavioral_1963}.
	
	Terlepas dari apakah penelitian tersebut akurat, penelitian tersebut menyebabkan stress emosional dan gangguan psikologis ekstrem yang dialami peserta penelitian.
	Walaupun Stanley Milgram menyebut bahwa tidak adak dampak negatif jangka panjang yang ditimbulkan, tetapi kritik menyebut penelitian tersebut melanggar etika penelitian dengan subyek manusia.
	
	Selain contoh tersebut, dampak negatif penelitian sosial dapat terjadi jika penelitian melanggar etika penelitian ilmiah.
	Misalnya, penelitian yang mengkaji topik yang kontroversial seperti misalnya prostitusi, dan peneliti tidak menyamarkan nama partisipan, hal ini secara tidak langsung dapat menimbulkan dampak negatif terhadap partisipan penelitian.
	Dampak-dampak negatif penelitian, baik yang telah diperkirakan maupun yang tidak diinginkan dapat diatasi dengan mematuhi etika penelitian ilmiah dan menerapkan langkah-langkah sebagai berikut \cite{sass_reichsrundschreiben_1983}.
	\begin{itemize}
		\item Adanya pemberitahuan persetujuan yang jelas dan tidak ambigu;
		\item Risiko harus diseimbangkan dengan manfaat dari penelitian;
		\item Perhatian khusus harus diambil untuk subyek yang berusia di bawah 18 tahun;
		\item Subyek dari golongan miskin dan tidak beruntung tidak boleh dieksploitasi demi kepentingan penelitian, ini termasuk iming-iming imbalan jika bersedia mengikuti penelitian;
		\item Jika memungkinkan, penelitian dengan subyek hewan dengan kecerdasan tinggi seperti kera dan simpanse didahulukan dibanding subyek manusia.
	\end{itemize}
\end{solution}

		
		\lipsum[3-7]
		
		\lipsum[2-4]
		% \input{soal/nama-file} % Ganti "nama-file" dengan nama file yang ingin ditambahkan
		%%%%%%%%%%%%%%%%%%%%%%%%%% dan hapus tanda persen
		
	\end{questions}
	
	\pagebreak
	\bibliographystyle{apacite} % atur gaya daftar pustaka, dalam hal ini menggunakan gaya APA
	\renewcommand{\refname}{Daftar Pustaka} % ubah nama daftar pustaka dari 'References' menjadi 
	%'Daftar Pustaka'
	\bibliography{ref} % Membuat daftar pustaka, "ref_pancasila" merupakan nama file bibtex yang 
	%berisi daftar pustaka
\end{document}
%%%%%%%%%%%%%%%%%%% INI ADALAH AKHIR KODE, JANGAN TAMBAHKAN APAPUN LAGI DIBAWAH %%%%%%%%%%%%%%%%%%%%%%