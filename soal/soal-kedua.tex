	\question
Dalam Ilmu-Ilmu Sosial khususnya Sosiologi, telah berkembang 2 pendekatan/metode penelitian, kuantitatif dan kualitatif. Carilah sumber rujukan ilmiah tentang kedua metode tersebut.
Berikan penjelasan tentang perbedaan landasan filosofis, tujuan dan prinsip-prinsip dalam penelitian sosial antara kedua pendekatan/metode tersebut; disertai dengan contoh yang relevan?
\begin{solution}
	Dalam penelitian ilmu sosial terdapat dua metode atau pendekatan yaitu penelitian kuantitatif dan kualitatif.
	Dua metode dalam penelitian ilmu sosial ini memiliki landasan filosofis, tujuan, dan prinsip-prinsip yang berbeda satu sama lain.
	
	Metode penelitian kuantitatif sendiri dipercaya merupakan pengaruh dari aliran filsafat positivisme \cite{ginting_filsafat_2008,mertens_research_2015,kothari_research_2013}.
	Filsafat positivisme sendiri didasarkan pada rasionalis dan empirisisme yang berdasarkan pada pemikiran Aristoteles, Auguste Comte, dan Immanuel Kant \cite{mertens_research_2015}.
	Metode kuantitatif menghendaki obyek yang teramati dan terukur untuk mencari hubungan atau pengaruh satu atau beberapa variabel terhadap variabel yang lain dengan menggunakan data-data yang bersifat empiris \cite{ginting_filsafat_2008}.
	Tujuan dalam penelitian kuantitatif yaitu `menjelaskan' \cite{ginting_filsafat_2008}.
	`Menjelaskan' yang di maksud adalah penelitian kuantitatif bertujuan untuk menjelaskan sebab-akibat satu atau lebih variabel dengan variabel yang lain berdasarkan fakta-fakta empiris. Sedangkan penelitian kualitatif memiliki tujuan `memahami'. Dengan kata lain, penelitian kualitatif bertujuan untuk memahami realitas yang ada dengan mengembangkan pengertian berdasarkan hasil kesimpulan yang diambil secara induktif.
	
	Metode penelitian kuantitatif memiliki prinsip-prinsip yang didasarkan pada aliran positivisme yaitu \cite{crossan_research_2003}:
	\begin{description}
		\item[Methodological] Semua penelitian harus menggunakan cara-cara yang terstruktur.
		\item[Value-freedom] Pengambilan keputusan dalam penelitian harus didasarkan pada kriteria objektif alih-alih kepercayaan dan kepentingan tertentu.
		\item[Causality] Tujuan penelitian haruslah untuk menemukan atau mengidentifikasi dan menjelaskan hubungan sebab-akibat dan hukum fundamental yang menjelaskan perilaku manusia.
		\item[Operationalisation] Konsep harus dibuat sedemikian rupa sehingga fakta dapat dianalisis dengan metode kuantitatif.
		\item[Independence] Peran peneliti independen dengan subjek yang diteliti.
		\item[Reductionism] Sebuah masalah akan lebih mudah dipahami jika masalah tersebut disederhanakan.
	\end{description}
	
	Penelitian kualitatif sendiri memiliki landasan filosofis pada konstruktivisme \cite{ginting_filsafat_2008,mertens_research_2015}.
	Penelitian kualitatif lahir dan berkembang dari tradisi ilmu-ilmu sosial di Jerman yang sarat diwarnai pemikiran filsafat ala platonik sebagaimana tercermin dalam pemikiran Kant maupun Hegel.
	Misalnya dalam pernyataan Hegel yang terkenal yakni `segala sesuatu yang riil adalah rasional dan segala sesuatu yang rasional adalah riil'. Paradigma penelitian kualitatif menuntut agar obyek yang diteliti untuk tidak dilepaskan dari konteksnya.
	Dengan kata lain metode penelitian kuantitatif bertolak belakang dengan metode penelitian kuantitatif dalam arti penelitian ini mengasumsikan bahwa realitas empiris terjadi bukan tanpa konteks sosio-kultural dan saling berkaitan satu sama lain.
	
	Sedangkan penelitian kualitatif memiliki prinsip-prinsip diantaranya.
	\begin{description}
		\item[Realitas Majemuk] Pendekatan kualitatif tidak hanya melihat realitas sebagai satu melainkan sebagai hal yang majemuk dan merupakan hasil dari konstruksi dalam pengertian holistik.
		\item[Interaktif] Berbeda dengan penelitian kuantitatif yang memandang bahwa peneliti dan subyek yang diteliti bersifat independen, dalam penelitian kualitatif melihat peneliti dan subyek penelitian sebagai bagian yang tidak terpisahkan.
		\item[Idiographic Statements] Penelitian kualitatif tidak lepas dari konteks waktu.
		\item[Tidak Bebas Nilai] Tidak seperti penelitian kuantitatif yang bebas nilai, penelitian kualitatif sebaliknya melihat sesuatu tidak bebas oleh nilai.
	\end{description}
	
	Salah satu contoh relevan penelitian kualitatif adalah dalam penelitian yang meneliti tentang transgender misalnya, menggunakan metode kualitatif membuat penelitian tersebut dilakukan dengan instrumen wawancara di mana peneliti terlibat dan berinteraksi langsung dengan subyek penelitian.
	
\end{solution}
