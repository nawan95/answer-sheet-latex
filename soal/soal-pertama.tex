\question
Carilah sumber rujukan tentang dua aliran dalam Filsafat Ilmu Sosial; (i) Positivisme-Realisme dan (ii) Konstruktivisme-Interpretivisme; yang melandasi arah dan perkembangan ilmu-ilmu sosial modern.
Berikan penjelasan tentang asumsi/pandangan tentang realitas sosial, fokus kajian dan implikasi dalam penelitian ilmiah dari kedua aliran tersebut?
\begin{solution}
	Aliran postivisme dan konstruktivisme adalah dua aliran dalam filsafat ilmu sosial yang sangat berbeda. Dalam hal pandangan tentang realitas sosial, positivisme memandang bahwa realitas sosial berjalan sesuai dengan hukum yang berlaku dan hanya melihat fakta-fakta empiris semata \cite{armstrong_positivism_2013}.
	Dengan kata lain, positivisme melihat realitas sosial tanpa melihat konteks sosial, motivasi, dan nilai yang ada di masyarakat yang mendasarinya.
	Dalam kasus misalnya fenomena bunuh diri, positivisme tidak melihat apa motivasi seseorang bunuh diri, tapi melihat penyebab bunuh diri dengan mencari fakta-fakta empiris kemudian disusun menjadi rantai logika hingga membentuk kausalitas; mayoritas kasus bunuh diri terjadi pada latar belakang sosial-ekonomi bawah, tidak memiliki pendidikan tinggi, pekerja serabutan, dan memiliki hutang.
	Postivisme kemudian mengambil kesimpulan bahwa bunuh diri disebabkan karena kondisi ekonomi dan lilitan hutang.
	Pendekatan positivisme berhenti di sini, ia tidak melihat konteks atau motivasi lain.
	Kenapa mereka miskin? Kenapa mereka terlilit hutang? Apa motivasi individu untuk mengakhiri hidup?
	Menurut aliran postivisme, hal itu tidak penting karena postivisme menolak spekulasi.
	Di sini dapat dilihat logika kausalitas dalam positivisme bersifat linear.
	Dengan kata lain, postivisme mengabaikan fakta bahwa realitas sosial sangat kompleks dan tidak bisa dipahami dengan logika kausalitas yang linier.
	Jika aliran positivisme melihat realitas sosial berjalan sesuai dengan hukum yang ada dan memiliki sebab-akibat yang pasti, konstruktivisme melihat bahwa realitas sosial bersifat subjektif, dan karenanya tidak ada fakta empiris dalam realitas sosial.
	
	Dalam hal fokus kajian, positivisme memiliki fokus kajian pada hasil-hasil observasi yang diinterpretasi dengan logika dan data-data empiris atau matematis dan menolak spekulasi-spekulasi teori dan hal-hal yang bersifat metafisik.
	Hal ini karena positivisme memandang bahwa informasi atau pengetahuan didapat melalui pengalaman sensori seperti observasi yang kemudian diinterpretasikan menggunakan akal dan logika \cite{macionis_sociology_2011}.
	Sedangkan konstruktivisme memiliki fokus kajian ada pada interaksi manusia dengan manusia yang lain dan dengan lingkungan.
	Hal ini karena konstruktivisme melihat pengetahuan merupakan hasil dari konstruksi oleh manusia dengan cara memberikan arti terhadap fakta atau data yang ada \cite{hinchey_chapter_2010}.
	Quantitative studies of social systems, underpinned by a positivist paradigm
	Seperti yang telah disinggung sebelumnya bahwa aliran positivisme dan konstruktivisme sangatlah berbeda.
	Pertanyaan yang muncul kemudian adalah apakah implikasi perbedaan konstruktivisme dan positivisme dalam penelitian ilmiah?
	Implikasi yang paling menonjol dari kedua aliran tersebut dalam penelitian ilmiah adalah metode penelitian yang digunakan.
	Metode penelitian kuantitatif didukung oleh aliran positivisme \cite{armstrong_positivism_2013}, sehingga penelitian ilmiah yang menggunakan paradigma positivisme cenderung menggunakan metode penelitian kuantitatif dengan analisis data alih-alih menggunakan metode kualitatif.
	Hal ini tidak mengherankan mengingat metode penelitian kuantitatif merupakan metode penelitian yang didasarkan pada pengumpulan dan analisis data-data yang bersifat empiris sesuai dengan aliran positivisme.
	Sedangkan penelitian ilmiah yang menggunakan pendekatan konstruktivisme cenderung menggunakan metode penelitian kualitatif \cite{mertens_research_2015}.
\end{solution}