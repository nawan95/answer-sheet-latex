	\question
Sed eu vestibulum velit. Vestibulum ante ipsum primis in faucibus orci luctus et ultrices posuere 
cubilia curae; Proin quis tortor et diam tempus fermentum ac in neque. Maecenas gravida a neque sit 
amet dictum. Quisque ultricies, turpis accumsan congue interdum, elit lorem rutrum est, quis rutrum 
purus ex maximus dolor?
\begin{solution}
	Salah satu ilustrasi dampak negatif dari sebuah penelitian sosial terhadap pihak-pihak yang terlibat dapat dilihat dari \textit{The Milgram experiment(s)}, sebuah penelitian psikologis sosial yang dilakukan oleh Stanley Milgram dari Universitas Yale pada 1961.
	Penelitian ini bertujuan untuk mencari tahu sampai sejauh mana orang-orang akan mematuhi figur yang memiliki otoritas ketika disuruh untuk melakukan hal yang berlawanan dengan hati nurani dan berbahaya \cite{milgram_behavioral_1963}.
	
	Terlepas dari apakah penelitian tersebut akurat, penelitian tersebut menyebabkan stress emosional dan gangguan psikologis ekstrem yang dialami peserta penelitian.
	Walaupun Stanley Milgram menyebut bahwa tidak adak dampak negatif jangka panjang yang ditimbulkan, tetapi kritik menyebut penelitian tersebut melanggar etika penelitian dengan subyek manusia.
	
	Selain contoh tersebut, dampak negatif penelitian sosial dapat terjadi jika penelitian melanggar etika penelitian ilmiah.
	Misalnya, penelitian yang mengkaji topik yang kontroversial seperti misalnya prostitusi, dan peneliti tidak menyamarkan nama partisipan, hal ini secara tidak langsung dapat menimbulkan dampak negatif terhadap partisipan penelitian.
	Dampak-dampak negatif penelitian, baik yang telah diperkirakan maupun yang tidak diinginkan dapat diatasi dengan mematuhi etika penelitian ilmiah dan menerapkan langkah-langkah sebagai berikut \cite{sass_reichsrundschreiben_1983}.
	\begin{itemize}
		\item Adanya pemberitahuan persetujuan yang jelas dan tidak ambigu;
		\item Risiko harus diseimbangkan dengan manfaat dari penelitian;
		\item Perhatian khusus harus diambil untuk subyek yang berusia di bawah 18 tahun;
		\item Subyek dari golongan miskin dan tidak beruntung tidak boleh dieksploitasi demi kepentingan penelitian, ini termasuk iming-iming imbalan jika bersedia mengikuti penelitian;
		\item Jika memungkinkan, penelitian dengan subyek hewan dengan kecerdasan tinggi seperti kera dan simpanse didahulukan dibanding subyek manusia.
	\end{itemize}
\end{solution}
